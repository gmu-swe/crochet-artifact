\title{CROCHET Artifact Guide}
\author{Jonhattan Bell \and Lu\'{i}s Pina}
\date{\today}

\documentclass[12pt]{article}

\usepackage{url}
\usepackage{hyperref}

\begin{document}
\maketitle

\begin{abstract}
This is the paper's abstract \ldots
\end{abstract}

\section{Introduction}

\subsection{Structure of this artifact}

This artifact consists of two VMs which are set-up and provisioned using
Vagrant.\footnote{\url{https://www.vagrantup.com/}}  You should install Vagrant
and Virtualbox in your system, we tested this artifact with Vagrant version XXX
and Virtualbox version XXX on GNU/Linux and Vagrant version YYY and Virtualbox
version YYY on MacOS.

\subsubsection{Main VM}

The first VM is the main VM that runs all the experiments except Section~XXX on
the paper.  It uses Ubuntu 16.04 LTS as the underlying operating system, and you
can find it on directory XXX.  It's structure is as follows:

\begin{description}

    \item[Vagrantfile] File that describes how Vagrant should provision the VM.
        Parameter \texttt{config.vm.provision}, at the end of the file,
        describes how to provision the artifact VM by running shell scripts
        listed below on this document.  Please customize the following
        parameters:

        \begin{description}

            \item[vm.memory] Memory that the VM will have.  We recommend at least 8G (8192MB).

            \item[vm.cpus] Number of CPUs available to the VM.  We recomment at least 2, more is better.

        \end{description}

    \item[downloads] The artifact requires some files to be downloaded from the
        internet and saved here.  We already populated this directory for your
        convenience.  File \texttt{scripts/downloads.sh} should be able to
        re-download them if needed.

        \begin{itemize}

            \item Oracle HotSpot JDK7

            \item Oracle HotSpot JDK8

            \item DaCapo 9.12 jar file

        \end{itemize}

    \item[scripts]  Several scripts and utilities used throughout the VM.  File
        \texttt{scripts/path.sh} is of particular interest as it sets the
        environment used throughout the artifact.  Ensure that variable
        \texttt{GLOBAL\_JVM\_PARAMS} configures JVMs with a maximum heap size
        that fits in the memory you configured the artifact VM with, in file
        \texttt{Vagrantfile} with parameter \texttt{vb.memory}.  We recommend at
        least 4GB with: \texttt{GLOBAL\_JVM\_PARAMS="-Xmx4G"}

    \item[experiments]  Each experiment reported in the paper has a
        subdirectory with the scripts to run it and generate a table with its
        results:

        \begin{description}

            \item[microbenchmarks] Section XXX

            \item[dacapo] Section XXX

            \item[dacapo-h2] Section XXX

            \item[ftp] Section XXX

            \item[stmbench7] Section XXX

        \end{description}

    \item[results]  The artifact saves the raw results of each experiment in
        this directory, under a subdirectory structure similar to the
        experiments structure.

    \item[tables]  The artifact saves in this directory the tables generated
        from the raw results.

\end{description}

\subsubsection{XJ VM}

The second VM only runs Section~XXX on the paper.

\section{Provision the VM}

\section{Run the experiments}

\subsection{Quick run}

\subsection{Paper run}

\section{Generate result tables}

\section{Run the artifact outside of the VM}

\end{document}
